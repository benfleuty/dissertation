Think of your dissertation as a story to be told to an interested reader (who could be considered as one of your peers in terms of subject knowledge and understanding) — in this way, a number of general points can be made:

The dissertation is not a diary and so need not follow the events in chronological order; choose an order that makes sense and is easy to follow.

You don't need to include everything. You may well leave out some work that took a lot of time or was very interesting, simply because you can't see how to include it in the main story (perhaps an Appendix).

Present work in a logical order. Give the reader the information they need to understand something, but give a gentle reminder if you presented it some way earlier in the dissertation.

Don't spend a lot of time describing or explaining something that doesn't get used or is pretty minor. The reader is putting effort into what they are reading and doesn't want to feel that any of that effort is wasted.

Don't try to explain everything to the ultimate degree. Use references to save yourself the trouble, but do explain things that are key to understanding your work. Going away to track down and check a reference should not be a necessity to understand the main aspects of the dissertation.

Relegate long pieces of explanation or extensive listings of data to an appendix; they would disrupt the reading of the dissertation if included in the main part.

The tone of the dissertation should be professional, honest, and upbeat. Don't dwell on problems or spend a lot of time making excuses — look to report things in a positive light.

10 000 words \pm 10\%
Excludes
\begin{itemize}
\item Abstract
\item ToC, LoF, LoT
\item Appendices
\item References
\end{itemize}